\documentclass{beamer}
\usetheme{Madrid} % or default, Copenhagen, etc.
\usecolortheme{beaver}
\usepackage{graphicx} % if adding images later

\title{Introduction to AI Agents and Large Language Models}
\subtitle{For Non-Experts}
\author{Daniel}
\date{\friday}

\begin{document}

\maketitle

\begin{frame}{What Are Large Language Models (LLMs)?}
  \begin{itemize}
    \item Very large AI trained on almost all text on the internet
    \item Core skill: predict the next word (extremely well)
    \item Result: can write emails, explain concepts, translate, write code, answer questions
  \end{itemize}
  \vspace{1em}
  \textbf{Examples everyone knows:}
  \begin{itemize}
    \item ChatGPT, Claude, Grok, Gemini
  \end{itemize}
  \vspace{1em}
  \small Like a super-smart autocomplete that “understands” a huge amount of knowledge.
\end{frame}

\begin{frame}{From Chatbot to Agent: The Big Jump}
  \begin{columns}
    \column{0.48\textwidth}
      \textbf{Normal LLM (Chat):}
      \begin{itemize}
        \item You ask a question
        \item It gives one answer
        \item Conversation ends (or continues linearly)
      \end{itemize}
    \column{0.48\textwidth}
      \textbf{AI Agent:}
      \begin{itemize}
        \item Has a goal
        \item Thinks step-by-step
        \item Uses tools (search, code, files, calculators…)
        \item Remembers previous steps
        \item Loops until goal achieved
      \end{itemize}
  \end{columns}
  \vspace{1em}
  \centering
  \Large Agent = LLM + Tools + Memory + Reasoning Loop
\end{frame}

\begin{frame}{How an AI Agent Actually Works}
  \begin{center}
    \textbf{Simple Loop Diagram}
  \end{center}
  \vspace{0.5em}
  \begin{enumerate}
    \item Get a clear goal (e.g., “Clean this messy CSV and summarize insights”)
    \item LLM thinks: “What do I know? What do I need to do first?”
    \item Choose action/tool (e.g., load file, check missing values)
    \item Execute → see result
    \item Think again: “Did that work? Next step?”
    \item Repeat until done → final output (report, cleaned data, etc.)
  \end{enumerate}
  \vspace{1em}
  \small This is often called ReAct (Reason + Act) pattern.
\end{frame}

\begin{frame}{Real-World Agent Examples (Simple & Useful)}
  \begin{itemize}
    \item Research agent: searches web, reads pages, writes summary report
    \item Data agent: loads CSV → finds problems → cleans → makes charts → explains insights
    \item Coding agent: writes, tests, fixes Python code for you
    \item Travel agent: checks flights, hotels, builds itinerary
  \end{itemize}
  \vspace{1em}
  \textbf{Today’s demo:} We build a data-cleaning \& EDA agent that works mostly by itself.
\end{frame}

\begin{frame}{Why Should Data Scientists Care?}
  \begin{block}{Agents = Your New Super Teammate}
    \begin{itemize}
      \item Automate boring / repetitive work (data wrangling, basic EDA)
      \item Handle multi-step problems without constant supervision
      \item Scale analysis: one person → many parallel agents
      \item Evolve fast: better LLMs → instantly smarter agents
    \end{itemize}
  \end{block}
  \vspace{1em}
  \centering
  \Large Limits exist (hallucinations, unsafe code, cost), but huge potential.
  \vfill
  \small Let’s build one right now!
\end{frame}

\end{document}
